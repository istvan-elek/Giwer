
\documentclass[a4paper,12pt]{article}
\usepackage[utf8]{inputenc}
\usepackage{graphicx}
\usepackage[magyar]{babel}
\usepackage{t1enc}

\begin{document}
	
	\author{Elek István}
	
	\title{Gyakorlás  \linebreak  \linebreak \linebreak \small DataStock, Catalog és WorkflowBuilder modulokhoz}
	
	
	\date{2022}
	
	
	\setcounter{tocdepth}{3}
	%\frontmatter
	\maketitle
	\newpage
	\tableofcontents
	\newpage
	%\mainmatter


\section{A Giwer keretrendszer}

Ebben a részben végigviszünk egy-egy feldolgozási folyamatot, megmutatjuk a különböző alrendszerek használatát a kiindulási adatoktól a végeredményig.

\section{DataStock gyakorlatok}

\subsection{Nyers adatok beolvasása és konverziója}

\subsection{Feldolgozási eljárások használata}

\section{Workflow builder gyakorlatok}

\section{Catalog gyakorlatok}

\end{document}
